We studied the relation between research productivity, as measured by citations and by centrality in the citations network, and academic salaries in the area of high-energy physics, in the University of California system.

Our evidence suggests even after controlling for the number of citations, years since Ph.D. is associated with higher and base salary. Furthermore, we find that the change in the number of citations is associated with increases in salary. One additional citation predicts an additional 33 USD in pay. In contrast, we found no effect on salary of total PageRank, $g$-index, or $h$-index under our model. The absence of an effect for the $g$- and $h$-indices suggests that in the academic labor market, citations are highly valued, even if they come from just a few highly-cited papers. The lack of effect of PageRank suggests that more complicated measures of centrality are probably not (yet) used for evaluating professors.

\subsection{Future work}
\paragraph{Robustness: expanding the dataset}
In the future, it will be useful to assess the robustness of these findings by obtaining wider datasets, such as other schools, and other areas of research (both other subfields of physics, and other fields). With data on multiple fields, it would be interesting to explore whether cross-field citations add as much value for a scholar’s salary as within-field citations. With regard to other schools, salary data can be found for several Texas university systems, the University of Massachusetts, the University of Michigan, the State University of New York, and several other public universities.

\paragraph{Controlling with different secondary regressors}
With additional data available, we could analyze the effect of different secondary variables. It would, for example, be interesting to study the effects on compensation of different school systems, states, genders (we did not have enough female professors in our UC data set to test this), and areas of research. For instance, gender salary equity is an issue faced by most colleges and universities~\cite{becker1995}. Further analysis could include different areas of research, location of schools, and teaching evaluations.

\paragraph{Other regression models}
In general, relations need not be linear. We did not attempt transforming the data (e.g. taking logarithms) or adding functions (e.g. square) of variables as regressors, because on observing plots of the data it was clear the obstacle to finding significance was noise rather than non-linearity of the relation. However, such techniques would be useful for exploration of further measures of research output. If a productivity measure is found to significantly predict salary, it would be interesting to test whether it offers \emph{incremental} explanatory power over the conventional measures (sum, $g$-, and $h$-indices of citation counts), using a multiple regression.

\paragraph{Individuals' contribution to papers with coauthors}
It also might be of interest to examine the contribution of each professor in a paper with coauthors. A paper in medicine or experimental high-energy physics might have thousands of citations, but also tens or hundreds of authors. One of the reasons why we limited our study to theory is that \textsc{hep-th} papers tend to have few coauthors. A study of coauthoring behavior across departments might be valuable for research universities.

\paragraph{Self-citations}
Often authors publish papers which build on their own previous work, and hence they cite their own previous papers. Evaluators of faculty often exclude such self-citations. It will be interesting to explore whether self-citations have any effect on faculty pay, and if so whether the effect is significantly different from other citations.

\paragraph{Weighted citations}
We have not examined the value of citations coming from different authors. Intuitively, a citation from a renowned professor would be a better indication of value of one\textquoteright{}s work, compared to a citation from an undergraduate student with little research experience. A model which assigns different weights to each citation, and also produces meaningful change over time is of interest.

\paragraph{Gamability}
Finally, measures of research productivity can potentially be gamed by scholars. For example, two friends could cite each other’s work, thereby mutually increasing their citation counts. Or further, there could form a social clique in which it is understood that members will cite each other’s work at the expense of outsiders. If (in future tests with larger datasets) more-sophisticated citation centrality measures turn out to be important determinants of pay, then it will be interesting to see whether new citations are especially likely to occur when such citations have a larger pecuniary benefit to the recipient, i.e., when the citation will increase the recipient’s centrality. It will also be interesting to test whether socially-linked scholars (e.g. who have been coauthors, are located at the same university or city, or graduated from the same university) are more likely to cite each other, and whether socially motivated citations are more likely when the centrality benefit to the recipient is higher.
