\subsection{Motivation}
According to economic theory, workers are paid according to productivity. In practice, employees are interested in knowing their productivities to know how much pay to expect, employers in measuring productivity to estimate the market price they must pay.

In most environments, it is difficult to measure productivity, because an individual\textquoteright{}s output is often combined with a large team\textquoteright{}s to yield the final product. However, in academia, citation networks provide the means to evaluate a scholar\textquoteright{}s research productivity. High-quality papers are expected to be cited more, and indices intended to quantify scientific productivity based on publication and citation record, like the $h$-index, are commonly calculated in services like Google Scholar to augment the summary of a professor\textquoteright{}s research output. Team memberships for each research project (coauthor sets) are explicitly listed, and coauthor sets differ across papers and are typically small, making it relatively easy to measure the productivity of individuals.

Citation counts and the $h$-index are largely based on the in-degrees of a scholar\textquoteright{}s papers, without regard to other features of network structure. This raises the question of whether we can construct a better measure of productivity that takes into account deeper features of the scholarly citation network. This parallels the question, in various other applications, of how to identify the importance of a node in a network. For example, search engines need to rank the importance of web pages for a keyword. For advertising and epidemic control, identifying the most important nodes in the network is a key issue as well. Hence, several measures of the centrality of nodes in a network have been developed.

\subsection{Previous research}
On the relation between academic salaries and research productivity, Grofman~\cite{grofman2009political}and Gomez-Mejia et al.~\cite{gomez1992} found a positive relation between citation counts and academic salary. Their main focus is on total citation count and citations per year as measures of research value.

However, citation counts are imperfect measures, for various reasons. The $h$-index was introduced to reflect a balance of quality and quantity in a scholar’s output. Its design also addresses the possible problem that a scholar’s citations might derive mainly from a very small number of papers. The $g$-index  (Egghe~\cite{egghe2006Gindex}) was designed to correct some drawbacks of the $h$-index.  

Furthermore, it is better to be cited by an important author than by a minor author. In consequence, Bollen et al.~\cite{bollen2006} argued that journal impact factors should incorporate centrality measurements like PageRank, which gives greater weight to citations from more important papers. PageRank is an eigenvector measure of centrality, thereby avoiding the problem of recursion in determining which citations are most important based on the importance of other citations. One contribution of our paper is to test the effect on pay of alternative productivity measures such as the $h$- and $g$-indices and more sophisticated measures of centrality like PageRank.

\subsection{Structural summary of productivity measures}
In this paper, we use a scholarly citation network and salary data from ten public universities to test whether academic faculty compensation increases with research productivity, which is considered to reflect both quantity and quality of output. Of course, research is not the sole determinant of faculty pay. Universities also reward faculty for teaching and administrative activities. Furthermore, pay may depend upon seniority, and upon geographical location (due to differing costs of living). Nevertheless, since research is an important output of faculty at research universities, we expect research performance to affect pay.

To construct a scholar’s score for a particular \textbf{centrality} measure and \textbf{aggregator} measure, we first construct a citation network for papers, and compute the \emph{centrality} of each paper in the network. We then \emph{aggregate} across a scholar’s papers to compute a centrality-based measure of the scholar’s productivity. We include traditional measures of scholarly output such as citation count and the $h$-index as special cases. Examples of centrality measures, aggregators, and productivity measures (combinations of the two) are shown in Table~\ref{tableTerminology}

\begin{table}[h]
	\centering
	\label{tableTerminology}
	\caption{Terminology: \emph{centrality measures} and \emph{aggregators}}
	
	\begin{tabular} {p{2.5cm}|p{2.5cm}||p{2cm}}
		\textbf{centrality measures}& \textbf{aggregators} & scholars' \textbf{productivity} measures \\
		 \textit{\footnotesize measure importance of a paper in the citation graph} &
		 \textit{\footnotesize combine the centralities of several papers into a single score measuring a scholar's research output}
		 & \\ \hline
		citation count & total ($\Sigma$) & $\Sigma(citations)$ \\
		betweenness & $g$-index & $g(citations)$ \\
		closeness & $h$-index & $h(citations)$ \\
		PageRank &  & $\Sigma(PageRank)$
	\end{tabular}
\end{table}

To address statistical issues, we estimate the relationship between changes in salary and changes in different productivity measures. We also control for other determinants of compensation, such as seniority. We hope this study will inspire consideration of alternative measures of productivity, and help scholars better understand how their pay is determined in the job market.

The remainder of the paper is organized as follows. First we introduce and motivate the general empirical model for studying the correlation between various measures of centrality and salary. We then describe the datasets that we employ. Following that we describe the specific measures chosen and regression models used, and present our results. Finally, we discuss the results and suggest topics for future research.
