\begin{abstract}
In academia, citation counts are often used to evaluate the quality of a professor\textquoteright~s research output. Since people expect to be compensated based on the value of their work, we should expect research output to predict salary. Using citation networks, we can quantify the productivity of a professor in several ways. We look at 18 years of citation networks in theoretical high-energy physics, and investigate whether seniority or several measures of productivity explain academic salary. We measure a scholar's productivity as a function (aggregation measure) of the centralities of each of his papers in the citation network. We find that seniority and higher total citation counts of scholars are associated with higher pay, and that an additional new citation on average increases pay by 33 USD. In contrast, more sophisticated measures of productivity ($h$- and $g$-indices of citation count, sum of PageRanks) are not associated with higher pay. We offer reasons why these measures do not predict salary. For further study we discuss possible biases and additional controls, and propose studying the "gamability" of productivity measures.
\end{abstract}
