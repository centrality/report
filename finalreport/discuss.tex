We can reject the null hypotheses and conclude that there exist relations between total citation count, citations per year, and $\Delta citations$ and salary. In contrast, we find no evidence of correlation with salary of the more sophisticated productivity measures.

\paragraph{Years since Ph.D. predicts salary}
Years since Ph.D. is statistically significantly correlated with base salary, and explains a substantial part of the variation in salary ($R^2 \approx 0.55$). This is unsurprising, since professors must demonstrate sustained productivity to qualify for promotions. Salary increases approximately $2400$ USD per year since Ph.D. on average.

\paragraph{New citations predict increases in salary}
Also, from the results of section~\secref{sectionDeltacit}, recent research output, as measured by new citations received, is positively associated with changes in salary. Specifically, each new citation translates to on average an additional $33$ USD in salary for high-energy physicists at UC.

\paragraph{PageRank not a significant predictor}
The coefficient on PageRank as a predictor of pay was statistically insignificant. A possible explanation is that eigenvector citation centrality is a sophisticated measure of performance; currently and historically, universities probably have not used PageRank as a measure of scholarly research output. Also, since PageRank is normalized so that PageRanks of all papers sum to $1$, as the number of papers increases over time, the pagerank of each paper decreases. Thus $\Delta(pagerank)$ is not useful. As a substitute for PageRank, we could estimate the importance of each citation by considering where the citing paper was published (in a top rank journal, a mid-range journal, a conference, etc.). But such a measure would be very subjective and hard to quantify, and is not studied here. In addition, if there are any missing nodes or edges in our sample, our estimates of PageRank will contain error. In general error in the independent variable will weaken its estimated relationship with the dependent variable. Borgatti, Kathleen M. Carley, David Krackhardt (2006) analyze the robustness of centrality measures to missing or spurious information.

\paragraph{$h(citations)$, $g(citations)$ not significant predictors}
The coefficients on the $h$- and $g$-indices in our tests are not statistically significant; we do not find that these measures are determinants of salary. A possible reason is that when a scholar has a few publications that are a very heavily cited, the h-index by design gives no incremental credit for such extreme home runs. This is arguably desirable, as a scholar might have a home-run paper by good luck, or by introducing a new methodology that everyone uses. However, universities may actually be willing to pay well for such highly influential papers, in which case total citations is a better predictor than the $h$-index of faculty pay. While the $g$-index gives more credit to highly-successful papers, both indices are still bounded by the number of papers, which can also cause under-weighting of home runs. For example, “[h]ad Albert Einstein died after publishing his four groundbreaking Annus Mirabilis papers in 1905, his h-index would be stuck at 4 or 5”~\cite{wikipediaEinstein}. Furthermore, both indices are almost completely insensitive to recent publications (which do not affect the $g$-/$h$-index at all until they have accumulated enough citations to place them in the author's top $g$/$h$ papers). As a result, our professors' calculated $g$- and $h$-indices change little over the period we have data for, leaving us insufficient data to regress on changes in the indices.

\subsection{General discussion}
We do not really expect to find a correlation between salary and the more sophisticated centrality measures. Since these measures are new, universities probably are not (yet) considering centrality when determining salaries, even if they are useful measures of productivity. It is possible that these measures are (or will be) considered while making decisions like hiring, promotions or giving awards. Current data shows no evidence of them being used for determining salary.

It is also of interest to compare the explanatory power of centrality and seniority on salary. Our results indicate that seniority has a strong effect on pay; each year since Ph.D. seems to add roughly $2400$ USD to base salary, while each new citation adds about $5$ USD. To compare the strengths of the effects requires common units. For example, we could examine the effect of a one standard deviation or one average year’s worth of increase. For high-energy physicists at UC, the effect of seniority apparently dominates that of citations, since few of our professors receive hundreds of citations per year.
